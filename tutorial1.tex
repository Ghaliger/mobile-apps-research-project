\documentclass[10pt,letterpaper]{article}
\begin{document}
\title{USING SMART PHONE APPLICATIONS TO ENHANCE EDUCATION IN UNIVERSITIES}
\maketitle
\section{ Introduction }
Every education institution and faculty staff strives to use anything that can benefit their students. However, its well-known that students oftentimes lack motivation and feel overwhelmed when tying to keep up with their numerous obligations associated with university life.
A great way to boost their motivation and inspire them is implementing their interests into education . Want them to do better and write more? Provide examples of a research paper and bring this assignment closer to them. At this point smart phones are an important part of their lives, they use all sorts of applications and having these platforms in education would make a brilliant impact. Every educator wants students to realize they can achieve their goals and dreams with hard wok, effort, and knowledge. That is why its highly imperative to take education to where students are and right now they are on their smart phones checking notifications, connecting with friends, reading on their tablets and so forth. Giving them access to lectures and class material through mobile applications makes perfect sense. This is a modern age and applications are the new frontier(Quillen, 2011); they have the power to revolutionize the way students learn.  
\section{Advantages of using mobile applications in revolutionizing education in Universities}

Smart phones make it easier for students to look up useful study material in fields they may not be familiar with, for example research in other facilities, or well-known organisations' data. When working on their assignments and essay for example, students are able to find sample research papers and templates as well as reference information.\\
In addition, mobile applications serve as a platform upon which class groups are created to send mass messages to students in order to pass on important educational material. By far the best example currently is Whatsapp, a social media platform that enables creation of agroup chat, flexible for a particular cause. Through these class groups, notes are passed to each member of the class as well as important announcements such as the schedules of lectures, deadlines of assignments, possible opportunities of say internship and so on.\\
The internet has  given the area of distance education digital steroids that have propelled online learning into a majour league status. Many universities are creating mobile applications to allow students to participate in their classmates to participate in their classwork in and outside their classrooms(Olavsrud, 2011).\\
Technology presented the human race with the miracle of  e-learning, and in so doing ruled out the "hussle" that came along with obtaining reading material especially text books( and my persona lfavourite, novels).Applications have been made, as a result, which not only  give you a variety of study knowledge free but also offer you a variety of formats to read this information in, such as e-pubs, pdfs, and so on. So, students are now able to obtain a library-full of books at a simple tap on the screen.
Mobile applications can be used with campus maps and GPS location to help students navigate across campuses as well as across directories and event schedules.\\
Professors and lecturers also benefit greatly from these mobile applications.Kukulska-Hulme and Traxler(as cited in Zawacki-Richter, Brown and Delport, 2009) state that mobile technologies can "open up new opportunities for independent investigation, practical fieldwork, professional updating, and on the spot access to knowledge. they also provide the mechanism for improved individual learner support and guidance." . Also, "Proffessors are able  able to utilize applications to send attendence reports, sennd automatic emails to absent stuudents and have class or group discussion forums."(Engebretson, 2010, para 3)\\

All in all, through these widely changing chanels of tecnology, students can maximize their effort, use plenty of resourses, improve writing skills with research paper templates, develope their their problem-solving skills, all of which prepares them for life.

The above research is basic, as its meant to look up the general knowledge about the use of mobile applications in education. It is also descriptive, because various angles of the topic were looked up descriptively to come up with this research. Finally, it is qualitative, because i barely at values in my analysis, but rather at the other said factors that i researched about.

\end{document}
